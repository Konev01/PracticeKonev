%% -*- coding: utf-8 -*-
\documentclass[12pt,a4paper]{scrartcl} 
\usepackage[utf8]{inputenc}
\usepackage[english,russian]{babel}
\usepackage{indentfirst}
\usepackage{misccorr}
\usepackage{graphicx}
\usepackage{amsmath}
\begin{document}
	\begin{titlepage}
		\begin{center}
			\large
			МИНИСТЕРСТВО НАУКИ И ВЫСШЕГО ОБРАЗОВАНИЯ РОССИЙСКОЙ ФЕДЕРАЦИИ
			
			Федеральное государственное бюджетное образовательное учреждение высшего образования
			
			\textbf{АДЫГЕЙСКИЙ ГОСУДАРСТВЕННЫЙ УНИВЕРСИТЕТ}
			\vspace{0.25cm}
			
			Инженерно-физический факультет
			
			Кафедра автоматизированных систем обработки информации и управления
			\vfill

			\vfill
			
			\textsc{Отчет по практике}\\[5mm]
			
			\LARGE\textit{Вариант 4}
			
			{\LARGE Решение системы линейных алгебраических уравнений методом Крамера}
			\bigskip
			
			1 курс, группа 1ИВТ АСОИУ
		\end{center}
		\vfill
		
		\newlength{\ML}
		\settowidth{\ML}{«\underline{\hspace{0.7cm}}» \underline{\hspace{2cm}}}
		\hfill\begin{minipage}{0.5\textwidth}
			Выполнил:\\
			\underline{\hspace{\ML}} Д.\,Е.~Конев\\
			«\underline{\hspace{0.7cm}}» \underline{\hspace{2cm}} 2024 г.
		\end{minipage}%
		\bigskip
		
		\hfill\begin{minipage}{0.5\textwidth}
			Руководитель:\\
			\underline{\hspace{\ML}} С.\,В.~Теплоухов\\
			«\underline{\hspace{0.7cm}}» \underline{\hspace{2cm}} 2024 г.
		\end{minipage}%
		
		
		\vfill
		
		
		
		\begin{center}
			
			Майкоп, 2024 г.
		\end{center}
	\end{titlepage}
\LARGE{Содержание}

\begin{enumerate}
	\item Задача
	\item Пример кода, решающего данную задачу
	\item Скриншот работы программы
\end{enumerate}
\section{Задача}
Решение системы линейных алгебраических уравнений методом Крамера (количество неизвестных меньше 4).
\section{Пример кода}
\label{sec:exp:code}
\begin{verbatim}
#include <iostream>
using namespace std;

int determinant(int matrix[3][3]);
int determinantX1(int coefMatrix[3][3],
 int constTermsMatrix[3]);
int determinantX2(int coefMatrix[3][3],
 int constTermsMatrix[3]);
int determinantX3(int coefMatrix[3][3],
 int constTermsMatrix[3]);

int main()
{
	int i, j;
	
	int coefficientsMatrix3x3[3][3];
	int constantTermsMatrix3x1[3];
	
	cout << "Vvedite koefficienty i 
	sbobodnye chleny " << endl;
	for (i = 0; i < 3; i++)
	{
		for (j = 0; j < 3; j++)
		{
			cout << "a[ " << i << "," << j << "]= ";
			cin >> coefficientsMatrix3x3[i][j];
		}
		cout << "b,[ " << i << "]= ";
		cin >> constantTermsMatrix3x1[i];
	}
	
	int det = determinant(coefficientsMatrix3x3);
	int detX1 = determinantX1(coefficientsMatrix3x3,
	 constantTermsMatrix3x1);
	int detX2 = determinantX2(coefficientsMatrix3x3,
	 constantTermsMatrix3x1);
	int detX3 = determinantX3(coefficientsMatrix3x3,
	 constantTermsMatrix3x1);
	
	if (det != 0)
	{
		cout << "X1 = " << (float)detX1 /
		 (float)det << endl;
		cout << "X2 = " << (float)detX2 /
		 (float)det << endl;
		cout << "X3 = " << (float)detX3 /
		 (float)det << endl;
	}
	else
	cout << "Sistema ne imejet reshenij " <<
	 endl << endl;
	
	return 0;
}

int determinant(int matrix[3][3])
{
	int a11 = matrix[0][0];
	int a12 = matrix[0][1];
	int a13 = matrix[0][2];
	int a21 = matrix[1][0];
	int a22 = matrix[1][1];
	int a23 = matrix[1][2];
	int a31 = matrix[2][0];
	int a32 = matrix[2][1];
	int a33 = matrix[2][2];
	
	return (a11 * a22 * a33) + (a12 * a23 * a31)
	 + (a13 * a21 * a32) -
	(a13 * a22 * a31) - (a11 * a23 * a32)
	 - (a12 * a21 * a33);
}

int determinantX1(int coefMatrix[3][3],
 int constTermsMatrix[3])
{
	int a12 = coefMatrix[0][1];
	int a13 = coefMatrix[0][2];
	int a22 = coefMatrix[1][1];
	int a23 = coefMatrix[1][2];
	int a32 = coefMatrix[2][1];
	int a33 = coefMatrix[2][2];
	int c1 = constTermsMatrix[0];
	int c2 = constTermsMatrix[1];
	int c3 = constTermsMatrix[2];
	
	return (c1 * a22 * a33) + 
	(a12 * a23 * c3) + (a13 * c2 * a32) -
	(a13 * a22 * c3) - (c1 * a23 * a32) 
	- (a12 * c2 * a33);
}

int determinantX2(int coefMatrix[3][3],
 int constTermsMatrix[3])
{
	int a11 = coefMatrix[0][0];
	int a13 = coefMatrix[0][2];
	int a21 = coefMatrix[1][0];
	int a23 = coefMatrix[1][2];
	int a31 = coefMatrix[2][0];
	int a33 = coefMatrix[2][2];
	int c1 = constTermsMatrix[0];
	int c2 = constTermsMatrix[1];
	int c3 = constTermsMatrix[2];
	
	return (a11 * c2 * a33) + (c1 * a23 * a31)
	 + (a13 * a21 * c3) -
	(a13 * c2 * a31) - (a11 * a23 * c3)
	 - (c1 * a21 * a33);
}

int determinantX3(int coefMatrix[3][3],
 int constTermsMatrix[3])
{
	int a11 = coefMatrix[0][0];
	int a12 = coefMatrix[0][1];
	int a21 = coefMatrix[1][0];
	int a22 = coefMatrix[1][1];
	int a31 = coefMatrix[2][0];
	int a32 = coefMatrix[2][1];
	int c1 = constTermsMatrix[0];
	int c2 = constTermsMatrix[1];
	int c3 = constTermsMatrix[2];
	
	return (a11 * a22 * c3) + (a12 * c2 * a31)
	 + (c1 * a21 * a32) -
	(c1 * a22 * a31) - (a11 * c2 * a32)
	 - (a12 * a21 * c3);
	 
	 
}
	
	
	
\end{verbatim}
\vfill


\section{Скриншот работы программы}
\label{sec:picexample}
\begin{figure}[h]
	\centering
	\includegraphics[width=0.9\textwidth]{practic.png}
	\caption{Результат}\label{fig:par}
\end{figure}
\end{document}